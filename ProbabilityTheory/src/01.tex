\documentclass[11pt,a4paper]{article}
\usepackage[utf8]{inputenc}
\usepackage[T2A]{fontenc}

\usepackage[russian]{babel}
\usepackage{setspace}
\usepackage{amsmath, amssymb, amsthm}
\usepackage[breaklinks]{hyperref}
\usepackage{xcolor}
\usepackage{lipsum}
\usepackage{fancyhdr}
\usepackage{geometry}
\usepackage{microtype}
\sloppy


\geometry{top=2cm, bottom=2cm, left=2cm, right=2cm}
\pagestyle{fancy}
\fancyhf{} 
\fancyhead[R]{\vspace{0.5em}4 сентября 2024\vspace{-0.5em}}


\setlength{\parindent}{0pt}
\setlength{\parskip}{1em}
\setlength{\textheight}{24cm}
\setlength{\textwidth}{16cm}
\setlength{\topmargin}{-5mm}
\setlength{\oddsidemargin}{0cm}
\setlength{\evensidemargin}{0cm}
\setstretch{1.3}
\singlespacing

\sloppy

\relpenalty=10000
\binoppenalty=10000

\theoremstyle{definition}
\newtheorem{definition}{Опр.}
\newtheorem{sv}{Свойство}
\newtheorem{sled}{Следствие}
\newtheorem{priz}{Признак}

\newenvironment{myproof}[1][\proofname]{
  \noindent \textbf{#1}
  \begin{flushleft}
    \hspace{-1em}$\square$
  \end{flushleft}
}{
  \hfill $\blacksquare$
}


\renewcommand{\thedefinition}{\arabic{definition}}




\title{Раздел I \\
       Случайные события}
\date{03.09.24}

\begin{document}
    \maketitle
    
    \textbf{Теория вероятностей} — наука, изучающая закономерности 
    в массовых случайных явлениях. Разделы курса: случайные события, случайные 
    величины, предельные теоремы.

    \section{Основные понятия и формулы}

    \begin{itemize}
        \item \(\Omega\) пространство элементарных исходов 
        \item \(\Omega = \{\omega_1, \omega_2, ... \omega_n\}\)
        множество всех единственно равновозможных исходов данного эксперимента
        \item \(F\) \(\sigma\)-алгебра случайных событий
        \item \(\prob\) вероятность
        \\
        \item \(A, B, C\dots\) случаные события
        \item \(U, \Omega\) достоверное событие
        \item \(V, \emptyset\) невозможное событие
        \item \(\bar A, \{\omega_i, \omega_i \notin A\}\) противоположное событие
    \end{itemize}
    \\
    Вероятность случайного события \(A\) определяем как \(P(A) = w_i, w_i \in [0,1]\).
    
    \begin{opr}
        \textbf{Полная группа событий}
        События \(H_{1}, \dots , H_{n}\) образуют полную группу событий, 
        если они попарно несовместны и среди них содержатся все элементарные исходы

        \begin{remark}
            Т.к. события полной группы несовместны, то вероятность их суммы равна 
            сумме их вероятностей, и при этом она равна единице:
            \begin{equation}
                P(H_{1}) + \dots + P(H_{n}) = 1
            \end{equation}
        \end{remark}
    \end{opr}


    \begin{opr}
        Условная вероятность \(P(A|B)\) - вероятность наступления \(A\) при условии, 
        что \(B\) наступило. Вычисляется как:
        \begin{equation}
            P(A|B) = \frac{P(AB)}{P(B)}
        \end{equation}
    \end{opr}


    \vspace{2cm}
    \begin{center}\large{10.09.24}\end{center}
    \begin{thm}
        \textbf{Формула произведения вероятностей}
        \begin{equation}
            P(A_{1}\dots A_{n}) = 
            P(A_{1})\cdot 
            P(A_{2}|A_{1})\cdot 
            P(A_{3} | P(A_{1} A_{2}))\cdot
            \dots
            P(A_{n} | A_{1}A_{2}\dots A_{n - 1})
        \end{equation}

        \begin{proof}
            TODO
        \end{proof}
    \end{thm}

    \begin{thm}
        \textbf{Формула полной вер-ти}
        Пусть \(H_{1}, H_{2}, \dots, H_{n}\) - полная группа событий. 
        \(A\) - вер-ть того, что произошло какое-то из них. Тогда
        \begin{equation}
            P(A) = \sum_{1}^{\infty} P(H_{i}) P(A|H_{i})
        \end{equation}
    \end{thm}


    \begin{thm}
        \textbf{Формула Байеса}
        Пусть \(H_{1}, H_{2}, \dots, H_{n}\) - полная группа событий. 
        Известно, что \(A\) произошло. Тогда:
        \begin{equation}
            P(H_{k} | A) = \frac{P(H_{k})\cdot P(A|H_{k})}{\sum_{1}^{\infty} P(H_{i}) P(A|H_{i})}
        \end{equation}
        \begin{proof}
            \begin{equation}
                P(H_{k} | A) = \frac{P(H_{k} A)}{P(A)} 
                = \frac{P(H_{k})\cdot P(A|H_{k})}{\sum_{1}^{\infty} P(H_{i}) P(A|H_{i})}
            \end{equation}
            (раскрыли усл. вер-ть по определению, в числ. применяем формулу произведения
            вероятностей, а в знаменателе — формулу полной вероятности)
        \end{proof}
    \end{thm}
    \section{Парадокс Монти-Холла}




    \vspace{2cm}
    \begin{center}\large{17.09.24}\end{center}
    \section{Схема независимых испытаний. Формула Бернулли}

    \begin{opr}
        \textbf{Независимые испытания} - испытания, исход каждого
        из которых не влияет на исход последующих
    \end{opr}

    \begin{opr}
        \textbf{Схема независимых испытаний Бернулли} - неоднократное
        вопроизведение независимых опытов в одинаковых условиях
    \end{opr}

    \begin{thm}
        \textbf{Формула Бернулли}: пусть проводится \(n\) незаивисимых одинаковых опытов, 
        в каждом из которых некоторое событие \(A\) может наступить с вероятностью \(p\) 
        и не наступить с вероятностью \(q\). Тогда вероятность того, что \(A\) наступит 
        ровно \(m\) раз (\(m \leq n\)):
        \begin{equation}
            P_{n, m}(A) = C^{m}_{n} p^{m} q^{n - m}
        \end{equation}

        \begin{proof}
            Пусть \(T\) - \(A\) наступило (\(P(T) = p\)), \(F\) - "\(A\) не наступило" 
            (\(P(T) = q\)), \(A^{'}\) - \( A \) наступило \(m\) раз из \(n\).\\
            Рассмотрим один из элементарных исходов, благоприятствующих 
            \(A^{'}\):
            \begin{equation}
                A_{1}^{'} = \underbrace{T\dots T}_{\text{\(m\) раз}}
                            \underbrace{F\dots F}_{\text{\(n - m\) раз}}
            \end{equation}
            т.к. испытания независимы, \(P(A_{1}^{'}) = p^{k}q^{n - k}\).\\
            Остальные элементарные исходы, благоприятствующие \(A^{'}\)
            отличаются только расстановкой событий \(T\) и \(F\). Количество таких 
            расстановок равно \(C^{m}_{n}\). Получаем
            \begin{equation}
                P_{n, m}(A) = C^{m}_{n} p^{m} q^{n - m}
            \end{equation}
        \end{proof}
    \end{thm}

    \section{Наивероятнейшее число успехов}
    \begin{opr}
        \(m^{*}\) - наивероятнейшее число успехов в данной серии
        испытаний относительно события \(A\), если при данном \(m\): \(P(A_{n, m})\to \max\)
    \end{opr}
    \begin{equation}
        
    \end{equation}
    \section{Геометрическое определение вероятности}
\end{document}
