\documentclass[11pt,a4paper]{article}
\usepackage[utf8]{inputenc}
\usepackage[T2A]{fontenc}

\usepackage[russian]{babel}
\usepackage{setspace}
\usepackage{amsmath, amssymb, amsthm}
\usepackage[breaklinks]{hyperref}
\usepackage{xcolor}
\usepackage{lipsum}
\usepackage{fancyhdr}
\usepackage{geometry}
\usepackage{microtype}
\sloppy


\geometry{top=2cm, bottom=2cm, left=2cm, right=2cm}
\pagestyle{fancy}
\fancyhf{} 
\fancyhead[R]{\vspace{0.5em}4 сентября 2024\vspace{-0.5em}}


\setlength{\parindent}{0pt}
\setlength{\parskip}{1em}
\setlength{\textheight}{24cm}
\setlength{\textwidth}{16cm}
\setlength{\topmargin}{-5mm}
\setlength{\oddsidemargin}{0cm}
\setlength{\evensidemargin}{0cm}
\setstretch{1.3}
\singlespacing

\sloppy

\relpenalty=10000
\binoppenalty=10000

\theoremstyle{definition}
\newtheorem{definition}{Опр.}
\newtheorem{sv}{Свойство}
\newtheorem{sled}{Следствие}
\newtheorem{priz}{Признак}

\newenvironment{myproof}[1][\proofname]{
  \noindent \textbf{#1}
  \begin{flushleft}
    \hspace{-1em}$\square$
  \end{flushleft}
}{
  \hfill $\blacksquare$
}


\renewcommand{\thedefinition}{\arabic{definition}}




\title{Intro}
\date{03.09.24}

\begin{document}
    \maketitle
    
    \textbf{Теория вероятностей} — наука, изучающая закономерности 
    в массовых случайных явлениях. Разделы курса: лучайные события, случайные 
    величины, предельные теоремы.

    \section{Основные понятия}

    \begin{itemize}
        \item \(\Omega\) пространство элементарных исходов 
        \item \(\Omega = \{\omega_1, \omega_2, ... \omega_n\}\)
        множество всех единственно равновозможных исходов данного эксперимента
        \item \(F\) \(\sigma\)-алгебра случайных событий
        \item \(\prob\) вероятность
        \\
        \item \(A, B, C\dots\) случаные события
        \item \(U, \Omega\) достоверное событие
        \item \(V, \emptyset\) невозможное событие
        \item \(\bar A, \{\omega_i, \omega_i \notin A\}\) противоположное событие
    \end{itemize}
    \\
    Вероятность случайного события \(A\) определяем как \(P(A) = w_i, w_i \in [0,1]\).
    
\end{document}
