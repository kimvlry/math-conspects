\documentclass[11pt,a4paper]{article}
\usepackage[utf8]{inputenc}
\usepackage[T2A]{fontenc}

\usepackage[russian]{babel}
\usepackage{setspace}
\usepackage{amsmath, amssymb, amsthm}
\usepackage[breaklinks]{hyperref}
\usepackage{xcolor}
\usepackage{lipsum}
\usepackage{fancyhdr}
\usepackage{geometry}
\usepackage{microtype}
\sloppy


\geometry{top=2cm, bottom=2cm, left=2cm, right=2cm}
\pagestyle{fancy}
\fancyhf{} 
\fancyhead[R]{\vspace{0.5em}4 сентября 2024\vspace{-0.5em}}


\setlength{\parindent}{0pt}
\setlength{\parskip}{1em}
\setlength{\textheight}{24cm}
\setlength{\textwidth}{16cm}
\setlength{\topmargin}{-5mm}
\setlength{\oddsidemargin}{0cm}
\setlength{\evensidemargin}{0cm}
\setstretch{1.3}
\singlespacing

\sloppy

\relpenalty=10000
\binoppenalty=10000

\theoremstyle{definition}
\newtheorem{definition}{Опр.}
\newtheorem{sv}{Свойство}
\newtheorem{sled}{Следствие}
\newtheorem{priz}{Признак}

\newenvironment{myproof}[1][\proofname]{
  \noindent \textbf{#1}
  \begin{flushleft}
    \hspace{-1em}$\square$
  \end{flushleft}
}{
  \hfill $\blacksquare$
}


\renewcommand{\thedefinition}{\arabic{definition}}




\title{Раздел II \\
       Случайные величины}
\date{17.09.24}

\begin{document}
    \maketitle
    \begin{opr}
        \textbf{Случайная величина} (СВ) - величина, которая может принимать
        все из своих возможных значений в зависимости от элементарных
        исходов испытания
        \\
        СВ обозначаются заглавными буквами, 
        а их значения - соответствующими маленькими буквами:
        \begin{equation}
            X, Y, X \dots . x, y, z
        \end{equation}
        \\
        Случайные величины: 
        \begin{itemize}
            \item \textbf{дискретные} (ДСВ) - возможные значения изолированы друг от друга 
                  (отдельные, конкретные значения, например - число, выпавшее при 
                  броске игральной кости)
            \item \textbf{непрерывные} (НСВ) - целиком заполняют собой определенный участок 
                  числовой прямой
                  \begin{opr} Строгое определение:
                    СВ \(X\) называется непрерывной, если для нее существует 
                    функция \(y = f(x)\) такая, что:
                    \begin{equation}
                        \forall x \in \R: \int_{-\infty}^{x} f(t)dt = F(x)
                    \end{equation}

                    \begin{ex}
                        \begin{equation}
                            f(x) = 0 \text{ при } x < 0, 
                            a\sin x \text{ при } 0\leq x \leq \pi, 
                            0 \text{ при } x > \pi
                        \end{equation}
                        \(a\) - ? \(F(x)\) - ? \(P(\frac{\pi}{3} < x < \frac{3\pi}{2})\) - ?
                        \begin{equation}
                            \begin{align*}
                                &\int_{-\infty}^{+\infty} f(x)dx = 1 \implies\\
                                &\int_{-\infty}^{0} 0dx + \int_{0}^{\pi} a\sin x dx + \int_{\pi}^{+\infty} 0 dx = \\
                                &-a\cos x |^{\pi}_{0} = -a(-1) - (-a) = 2a = 1 \implies\\
                                &a = \frac{1}{2}
                            \end{align*}
                        \end{equation}
                    \end{ex}
                  \end{opr}
        \end{itemize}
        \begin{ex}
            Ошибка при взвешивания вещества - это НСВ, кол-во ошибок - это ДСВ. 
        \end{ex}
    \end{opr}

    \begin{opr}
        \textbf{Закон распределения} - соотношение между
        всевозможными значениями СВ и их вероятностями
    \end{opr}



    \vspace{2cm}
    \begin{center}\large{24.09.24}\end{center}
    \section{Закон распределения ДСВ}
    \begin{ex}
        В ящике 2 белых, 8 черных шаров. Одновременно достают 3.\\
        X - число белых среди них. 
    \end{ex}

    \subsection{Ряд распределения}
        \begin{equation}
            \begin{align*}
                &x_{0} = 0 ; &m_{1} = \\
                &x_{1} = 1 ; &m_{2} = \\
                &x_{2} = 2 ; &m_{3} = \\
            \end{align*}

            \begin{align*}
                &x_{i} &0 &1 &2\\
                &p_{i} &\frac{56}{120} &\frac{56}{120} &\frac{8}{120}
            \end{align*}

            \begin{tabular}{|c|c|c|}
                
            \end{tabular}
        \end{equation}

    \subsection{Многоугольник распределения}

    \subsection{Функция распределения}    
        вероятность того, что \(X\) примет определенное значение относительно \(x\).
        \begin{equation}
            F(x) = P(X < x)
        \end{equation}
        \begin{ex}
            \begin{equation}
                F(-1) = P(X < -1)
            \end{equation}
        \end{ex}

        \begin{sv}
            Функция распределения везде определена: \(D(F) = (-\infty; +\infty)\)
        \end{sv}
        \begin{sv}
            Область значений совпадает с областью значений вероятности: \(E(F) = [0; 1]\)
        \end{sv}
        \begin{sv}
            \lim_{x\to -\infty} F(x) = 0\\
            \lim_{x\to +\infty} F(x) = 1
        \end{sv}
        \begin{sv}
            Функция распределения неубывающая
        \end{sv}
        \begin{sv}
            Имеет разрывы I рода в точках \(x = x_{i}\) и величина разрыва равна \(p_{i}\) 
            (разрывы отсчитываются от предыдущего уровня, а не от \(y = 0\))
        \end{sv}
        \begin{sv}
            \(P(a \leq x < b) = F(b) - F(a)\)\\
            \begin{proof}
                \begin{equation}
                    \begin{align*}
                        &A. x < a\\
                        &B. x < b\\
                        &C. a \leq x < b\\
                        &A + C = B, \text{а также } A, C \text{ несовместны}\\
                        &P(A + C) = P(A) + P(C)\\
                        \\
                        &P(A) + P(C) = P(B) \implies P(C) = P(B) - P(A) = \\
                        & P(x < b) - P(x < a) = F(b) - F(a)
                    \end{align*}
                \end{equation}
            \end{proof}
        \end{sv}
    

        \section{Закон распределения НСВ}
        \begin{equation}
            P(x = x_{i}) = 0.
        \end{equation}
        \(A\) - невозможное \(\implies P(A) = 0\). 
        \begin{remark}
            не работает в обратную сторону! 
        \end{remark}

        \begin{sv}
            \begin{equation}
                F(X) = P(X < x)
            \end{equation}
        \end{sv}




    \vspace{2cm}
    \begin{center}\large{01.10.24}\end{center}

    \section{Числовые характеристики случайной величины}
    \begin{opr}
        \textbf{Математическое ожидание} (\(M(x); E(x); M\xi\))
        - среднее по вероятности значение случайной величины. Для ДСВ:
        \begin{equation}
            M(x) = \frac{\sum x_{i}p_{i}}{\sum p_{i}} = \sum_{i = 1}^{n} x_{i} p_{i}
        \end{equation}
        Физический смысл:
        бр бр бррр

        Переход к формуле мат. ожидания для НСВ: \(x_{i}\to x; p_{i}\to f(x)dx; \sum_{i = 1}^{n} \to \int_{-\infty}^{+\infty}\)
        \begin{equation}
            &M(x) = \int_{-\infty}^{+\infty} x f(x)dx
        \end{equation}
        геом. смысл: соответсвует абсциссе центра тяжести графика фигуры, образованной графиком \(f(x)\)
    \end{opr}

    \begin{opr}
        \textbf{Мода} \(m_{o}\) - наиболее возможное значение случайной величины 
    \end{opr}

    \begin{opr}
        \textbf{Медиана} \(m_{e} \)- такое значение, что вероятность того, что 
        \begin{equation}
            \(P(x < m_{e}) = P(x > m_{e}) = \frac{1}{2}\)
        \end{equation} 
        геом. смысл: точка, которая делит площадь под графиком плотности пополам.

        \begin{remark}
            Если распределение дискретное и не симметричное, то за \(m_{e}\) принимают
            такое значение СВ, что сумма вероятностей до него и после него минимально
            отличаются друг от друга.
        \end{remark}
    \end{opr}

    \begin{opr}
        \textbf{Дисперсия} \(D(x)\) - степень отклонения значений СВ от мат. ожидания. 
        Или: степень разброса значений СВ относительно мат. ожидания.
        \begin{equation}
            D(x) = M((x - M(x))^{2})
        \end{equation}

        Для ДСВ:
        \begin{equation}
            D(x) = \sum_{i = 1}^{n}(x_{i} - M(x))^{2}p_{i}
        \end{equation}
        \begin{equation}
            \begin{align*}
                &(x - M(x))^{2} = x^{2} - 2xM(x) + (M(x))^{2}\\
                &M(x^{2} - 2xM(x) + (M(x))^{2}) = M(x^{2}) - 2M(x)M(x) + (M(x))^{2} = \\
                &M(x^{2}) - 2(M(x))^{2} + (M(x))^{2} = \\
                &M(x^{2}) - (M(x))^{2}
            \end{align*}
        \end{equation}

        Для НСВ:
        \begin{equation}
            D(x) = \int_{-\infty}^{+\infty}(x_{i} - M(x))^{2}f(x)dx
        \end{equation}
    \end{opr}

    \begin{ex}
        \begin{equation}
            \begin{align*}
                &x_{i} = : 0; 1; 2\\
                &p_{i} = : \frac{7}{15}; \frac{7}{15}; \frac{1}{15}
            \end{align*}
        \end{equation}
        для ДСВ:
        \begin{equation}
            \begin{align*}
                &M(x) = 0\cdot \frac{7}{15} + 1\cdot \frac{7}{15} + 2\cdot \frac{1}{15} = \frac{9}{15}\\
                &m_{o} = 0 \wedge m_{o} = 1 \text{ (две моды)}\\
                &m_{e} = 1\\
                & \\
                & \\
                &D(x) = 0^{2}\cdot \frac{7}{15} + 1^{2}\cdot \frac{7}{15} + 2^{2}\cdot \frac{1}{15} - {\frac{9}{15}}^{2} \\
                & = \frac{7}{15} + \frac{4}{15} - \frac{81}{225}\\
                & = \frac{28}{75}
            \end{align*}
        \end{equation}
        для НСВ:
        \begin{equation}
            f(x) = 
            \begin{cases}
                0, x < 0 \\
                \frac{1}{2}\sin x; 0 \leq x \geq \pi \\
                0, x > \pi
            \end{cases}

            \begin{align*}
                &M(x) = \\
                & = \frac{\pi}{2} \\
                & \\
                & P(x < m_{e}) = \frac{1}{2} \text{  ясно, что \(m_{e} \in [0; \pi]\)}\\
                & P(x < m_{e}) = \int_{-\infty}^{m_{e}} f(x)dx = \int_{-\infty}^{0} 0dx + \int_{0}^{m_{e}} \frac{1}{2}\sin xdx = \frac{1}{2}\\
                & \implies \int_{0}^{m_{e}} \frac{1}{2}\sin xdx = \frac{1}{2} \dots\\
                & \dots = \frac{\pi}{2}\\
                &\\
                &\\
                & D(x) = \int_{-\infty}^{+\infty} x^{2}f(x)dx - (M(x))^{2}\\
                & = \int_{0}^{\pi} x^{2} \frac{1}{2}\sin xdx - {\frac{\pi}{2}}^{2}\\
                & = \frac{1}{2} \int_{0}^{\pi} x^{2} \sin xdx - {\frac{\pi}{2}}^{2} \\
                & = \dots = 
            \end{align*}
        \end{equation}
    \end{ex}

    \begin{opr}
        \textbf{Среднеквадратическое отклонение} (СКО) \(\sigma(x)\)
        \begin{equation}
            \sigma(x) = \sqrt{D(x)}
        \end{equation}
        \begin{equation}
            \begin{align*}
                &[M(x)] = [m_{o}] = [m_{e}] = [x]\\
                &[D(x)] = [x^{2}]\\
                &[\sigma(x)] = [x]
            \end{align*}
        \end{equation}
        имеет ту же размерность, что и мат. ожидание 
    \end{opr}

    \subsection{Основные свойства мат. ожидания\(M(x)\)}
    \begin{sv}
        \(M(C) = C\)
    \end{sv}
    \begin{sv}
        \(M(x\cdot C) = C\cdot M(x)\)
        \begin{proof}
            
        \end{proof}
    \end{sv}
    \begin{sv}
        \(M(x +- y) = M(x) +- M(y)\)
        \begin{proof}
            
        \end{proof}
    \end{sv}
    \begin{sv}
        если \(x, y\) независимы: \(M(x\cdot y) = M(x)\cdot M(y)\)
    \end{sv}


    \subsection{Основные свойства дисперсии \(D(x)\)}
    \begin{sv}
        \(D(C) = 0\)
    \end{sv}
    \begin{sv}
        \(D(s\cdot C) = C^{2} D(x)\)
        \begin{proof}
            \begin{equation}
                \begin{align*}
                    &D(x\cdot C) \\
                    &= M((xC - M(xC)))^{2} \\
                    &= M((xC - CM(x)))^{2} \\
                    &= M(C^{2}(x - M(x))^{2}) \\
                    &= C^{2}M((x - M(x)))^{2} \\
                    &= C^{2}D(x)
                \end{align*}
            \end{equation}
        \end{proof}
    \end{sv}
    \begin{sv}
        если \(x, y\) независимы: \(D(x +- y) = D(x) +- D(y)\)
    \end{sv}



    \subsection{Понятие о начальных и центральных теоретических моментах}
    \begin{opr}
        \textbf{Начальный момент порядка k}
        \begin{equation}
            \nu_{k} = M(x^{k}) = 
            \begin{cases}
                \sum_{i = 1}^{n} x_{i}p_{i} \text{  для ДСВ}\\
                \int_{-\infty}^{+\infty} x^{k} f(x)dx \text{  для НСВ}
            \end{cases}
        \end{equation}
    \end{opr}

    \begin{opr}
        \textbf{Центральный момент порядка k}
        \begin{equation}
            \mu_{k} = M((x - M(x))^{k}) = 
            \begin{cases}
                \sum_{i = 1}^{n} (x_{i} - M(x))^{k} \text{  для ДСВ}\\
                \int_{-\infty}^{+\infty} (x - M(x))^{k}f(x)dx \text{  для НСВ}
            \end{cases}
        \end{equation}
    \end{opr}

    \begin{ex}
        \begin{equation}
            \begin{align*}
                &\mu_{2} = D(x) = \nu_{2} - (\nu_{1})^{2}\\
                &D(x) = \sum_{i=1}^{n}x_{i}p_{i} - (M(x))^{2} = \nu_{2} - (\nu_{1})^{2}
            \end{align*}
        \end{equation}
    \end{ex}


    \begin{opr}
        \textbf{Коэффициент асимментрии} - характеризует меру скошенности распределения
        \begin{equation}
            A = \frac{\mu_{3}}{\sigma^{3}}
        \end{equation}
    \end{opr}
    \begin{opr}
        \textbf{Коэффициент эксцесса} - степень "пологости". 
        \(E_{k} > 0\) - "заостренное", \(E_{k} < 0\) - "пологое"
        \begin{equation}
            E_{k} = \frac{\mu_{4}}{\sigma^{4}} - 3
        \end{equation}
    \end{opr}



    \section{Основные законы распределения случайных величин}
    \subsection{Дискретные}
    \begin{opr}
        \textbf{Равномерное} - все вероятности одинаковы 
        \begin{equation}
            p_{i} = \frac{1}{n}
        \end{equation}
        при этом 
        \begin{equation}
            \begin{align*}
                &M(x) = \frac{n + 1}{2}\\
                &D(x) = \frac{n^{2}-1}{12}
            \end{align*}
        \end{equation}
        \begin{ex}
            бросание игральной кости
        \end{ex}
    \end{opr}
\end{document}