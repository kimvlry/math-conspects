\documentclass[11pt,a4paper]{article}
\usepackage[utf8]{inputenc}
\usepackage[T2A]{fontenc}

\usepackage[russian]{babel}
\usepackage{setspace}
\usepackage{amsmath, amssymb, amsthm}
\usepackage[breaklinks]{hyperref}
\usepackage{xcolor}
\usepackage{lipsum}
\usepackage{fancyhdr}
\usepackage{geometry}
\usepackage{microtype}
\sloppy


\geometry{top=2cm, bottom=2cm, left=2cm, right=2cm}
\pagestyle{fancy}
\fancyhf{} 
\fancyhead[R]{\vspace{0.5em}4 сентября 2024\vspace{-0.5em}}


\setlength{\parindent}{0pt}
\setlength{\parskip}{1em}
\setlength{\textheight}{24cm}
\setlength{\textwidth}{16cm}
\setlength{\topmargin}{-5mm}
\setlength{\oddsidemargin}{0cm}
\setlength{\evensidemargin}{0cm}
\setstretch{1.3}
\singlespacing

\sloppy

\relpenalty=10000
\binoppenalty=10000

\theoremstyle{definition}
\newtheorem{definition}{Опр.}
\newtheorem{sv}{Свойство}
\newtheorem{sled}{Следствие}
\newtheorem{priz}{Признак}

\newenvironment{myproof}[1][\proofname]{
  \noindent \textbf{#1}
  \begin{flushleft}
    \hspace{-1em}$\square$
  \end{flushleft}
}{
  \hfill $\blacksquare$
}


\renewcommand{\thedefinition}{\arabic{definition}}




\title{Раздел II \\
       Случайные величины}
\date{17.09.24}

\begin{document}
    \maketitle
    \begin{opr}
        \textbf{Случайная величина} (СВ) - величина, которая может принимать
        все из своих возможных значений в зависимости от элементарных
        исходов испытания
        \\
        \\
        Случайные величины: 
        \begin{itemize}
            \item дискретные (ДСВ) - возможные значения изолированы друг от друга 
                  (отдельные, конкретные значения, например - число, выпавшее при 
                  броске игральной кости)
            \item непрерывные (НСВ) - целиком заполняют собой определенный участок 
                  числовой прямой
        \end{itemize}
        \begin{ex}
            Ошибка при взвешивания вещества - это НСВ, кол-во ошибок - это ДСВ. 
        \end{ex}
    \end{opr}

    \\
    \\
    \begin{opr}
        \textbf{Закон распределения} - соотношение между
        всевозможными значениями СВ и их вероятностями
    \end{opr}



    \vspace{2cm}
    \centering\large{24.09.24}
    
\end{document}