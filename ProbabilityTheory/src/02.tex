\documentclass[11pt,a4paper]{article}
\usepackage[utf8]{inputenc}
\usepackage[T2A]{fontenc}

\usepackage[russian]{babel}
\usepackage{setspace}
\usepackage{amsmath, amssymb, amsthm}
\usepackage[breaklinks]{hyperref}
\usepackage{xcolor}
\usepackage{lipsum}
\usepackage{fancyhdr}
\usepackage{geometry}
\usepackage{microtype}
\sloppy


\geometry{top=2cm, bottom=2cm, left=2cm, right=2cm}
\pagestyle{fancy}
\fancyhf{} 
\fancyhead[R]{\vspace{0.5em}4 сентября 2024\vspace{-0.5em}}


\setlength{\parindent}{0pt}
\setlength{\parskip}{1em}
\setlength{\textheight}{24cm}
\setlength{\textwidth}{16cm}
\setlength{\topmargin}{-5mm}
\setlength{\oddsidemargin}{0cm}
\setlength{\evensidemargin}{0cm}
\setstretch{1.3}
\singlespacing

\sloppy

\relpenalty=10000
\binoppenalty=10000

\theoremstyle{definition}
\newtheorem{definition}{Опр.}
\newtheorem{sv}{Свойство}
\newtheorem{sled}{Следствие}
\newtheorem{priz}{Признак}

\newenvironment{myproof}[1][\proofname]{
  \noindent \textbf{#1}
  \begin{flushleft}
    \hspace{-1em}$\square$
  \end{flushleft}
}{
  \hfill $\blacksquare$
}


\renewcommand{\thedefinition}{\arabic{definition}}




\title{Раздел II \\
       Случайные величины}
\date{17.09.24}

\begin{document}
    \maketitle
    \begin{opr}
        \textbf{Случайная величина} (СВ) - величина, которая может принимать
        все из своих возможных значений в зависимости от элементарных
        исходов испытания
        \\
        СВ обозначаются заглавными буквами, 
        а их значения - соответствующими маленькими буквами:
        \begin{equation}
            X, Y, X \dots . x, y, z
        \end{equation}
        \\
        Случайные величины: 
        \begin{itemize}
            \item \textbf{дискретные} (ДСВ) - возможные значения изолированы друг от друга 
                  (отдельные, конкретные значения, например - число, выпавшее при 
                  броске игральной кости)
            \item \textbf{непрерывные} (НСВ) - целиком заполняют собой определенный участок 
                  числовой прямой
                  \begin{opr} Строгое определение:
                    СВ \(X\) называется непрерывной, если для нее существует 
                    функция \(y = f(x)\) такая, что:
                    \begin{equation}
                        \forall x \in \R: \int_{-\infty}^{x} f(t)dt = F(x)
                    \end{equation}

                    \begin{ex}
                        \begin{equation}
                            f(x) = 0 \text{ при } x < 0, 
                            a\sin x \text{ при } 0\leq x \leq \pi, 
                            0 \text{ при } x > \pi
                        \end{equation}
                        \(a\) - ? \(F(x)\) - ? \(P(\frac{\pi}{3} < x < \frac{3\pi}{2})\) - ?
                        \begin{equation}
                            \begin{align*}
                                &\int_{-\infty}^{+\infty} f(x)dx = 1 \implies\\
                                &\int_{-\infty}^{0} 0dx + \int_{0}^{\pi} a\sin x dx + \int_{\pi}^{+\infty} 0 dx = \\
                                &-a\cos x |^{\pi}_{0} = -a(-1) - (-a) = 2a = 1 \implies\\
                                &a = \frac{1}{2}
                            \end{align*}
                        \end{equation}
                    \end{ex}
                  \end{opr}
        \end{itemize}
        \begin{ex}
            Ошибка при взвешивания вещества - это НСВ, кол-во ошибок - это ДСВ. 
        \end{ex}
    \end{opr}

    \begin{opr}
        \textbf{Закон распределения} - соотношение между
        всевозможными значениями СВ и их вероятностями
    \end{opr}



    \vspace{2cm}
    \begin{center}\large{24.09.24}\end{center}
    \section{Закон распределения ДСВ}
    \begin{ex}
        В ящике 2 белых, 8 черных шаров. Одновременно достают 3.\\
        X - число белых среди них. 
    \end{ex}

    \subsection{Ряд распределения}
        \begin{equation}
            \begin{align*}
                &x_{0} = 0 ; &m_{1} = \\
                &x_{1} = 1 ; &m_{2} = \\
                &x_{2} = 2 ; &m_{3} = \\
            \end{align*}

            \begin{align*}
                &x_{i} &0 &1 &2\\
                &p_{i} &\frac{56}{120} &\frac{56}{120} &\frac{8}{120}
            \end{align*}
        \end{equation}

    \subsection{Многоугольник распределения}

    \subsection{Функция распределения}    
        вероятность того, что \(X\) примет определенное значение относительно \(x\).
        \begin{equation}
            F(x) = P(X < x)
        \end{equation}
        \begin{ex}
            \begin{equation}
                F(-1) = P(X < -1)
            \end{equation}
        \end{ex}

        \begin{sv}
            Функция распределения везде определена: \(D(F) = (-\infty; +\infty)\)
        \end{sv}
        \begin{sv}
            Область значений совпадает с областью значений вероятности: \(E(F) = [0; 1]\)
        \end{sv}
        \begin{sv}
            \lim_{x\to -\infty} F(x) = 0\\
            \lim_{x\to +\infty} F(x) = 1
        \end{sv}
        \begin{sv}
            Функция распределения неубывающая
        \end{sv}
        \begin{sv}
            Имеет разрывы I рода в точках \(x = x_{i}\) и величина разрыва равна \(p_{i}\) 
            (разрывы отсчитываются от предыдущего уровня, а не от \(y = 0\))
        \end{sv}
        \begin{sv}
            \(P(a \leq x < b) = F(b) - F(a)\)\\
            \begin{proof}
                \begin{equation}
                    \begin{align*}
                        &A. x < a\\
                        &B. x < b\\
                        &C. a \leq x < b\\
                        &A + C = B, \text{а также } A, C \text{ несовместны}\\
                        &P(A + C) = P(A) + P(C)\\
                        \\
                        &P(A) + P(C) = P(B) \implies P(C) = P(B) - P(A) = \\
                        & P(x < b) - P(x < a) = F(b) - F(a)
                    \end{align*}
                \end{equation}
            \end{proof}
        \end{sv}
    

        \section{Закон распределения НСВ}
        \begin{equation}
            P(x = x_{i}) = 0.
        \end{equation}
        \(A\) - невозможное \(\implies P(A) = 0\). 
        \begin{remark}
            не работает в обратную сторону! 
        \end{remark}

        \begin{sv}
            \begin{equation}
                F(X) = P(X < x)
            \end{equation}
        \end{sv}
\end{document}