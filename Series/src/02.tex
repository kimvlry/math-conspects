\documentclass[11pt,a4paper]{article}
\usepackage[utf8]{inputenc}
\usepackage[T2A]{fontenc}

\usepackage[russian]{babel}
\usepackage{setspace}
\usepackage{amsmath, amssymb, amsthm}
\usepackage[breaklinks]{hyperref}
\usepackage{xcolor}
\usepackage{lipsum}
\usepackage{fancyhdr}
\usepackage{geometry}
\usepackage{microtype}
\sloppy


\geometry{top=2cm, bottom=2cm, left=2cm, right=2cm}
\pagestyle{fancy}
\fancyhf{} 
\fancyhead[R]{\vspace{0.5em}4 сентября 2024\vspace{-0.5em}}


\setlength{\parindent}{0pt}
\setlength{\parskip}{1em}
\setlength{\textheight}{24cm}
\setlength{\textwidth}{16cm}
\setlength{\topmargin}{-5mm}
\setlength{\oddsidemargin}{0cm}
\setlength{\evensidemargin}{0cm}
\setstretch{1.3}
\singlespacing

\sloppy

\relpenalty=10000
\binoppenalty=10000

\theoremstyle{definition}
\newtheorem{definition}{Опр.}
\newtheorem{sv}{Свойство}
\newtheorem{sled}{Следствие}
\newtheorem{priz}{Признак}

\newenvironment{myproof}[1][\proofname]{
  \noindent \textbf{#1}
  \begin{flushleft}
    \hspace{-1em}$\square$
  \end{flushleft}
}{
  \hfill $\blacksquare$
}


\renewcommand{\thedefinition}{\arabic{definition}}




\title{Знакопеременные и знакочередующиеся ряды}
\date{18.09.24}

\begin{document}
\maketitle
\begin{remark}
    Знакочередующимся называется ряд, где знак меняется 
    для каждого следующего члена ряда. В знакопеременном ряде 
    знак может изменяться по любому правилу.
\end{remark}

\begin{thm}
    Признак Лейбница (о сходимости знакочередующегося ряда)\\
    Дан знакочередующийся ряд \(\sum_{n=1}^{\infty} (-1)^{n-1} U_{n}\). 
    Если последоавтельность модулей членов ряда монотонно убывает 
    и \(\lim_{n\to\infty} U_{n}=0\), то ряд сходится.
    \begin{proof}
        Рассмотрим частичную сумму \(S_{2k}\):
        \begin{equation}
            S_{2k} = (U_{1} - U_{2}) + (U_{3} - U_{4}) + 
            \dots + (U_{2k - 1} - U_{2k})
        \end{equation}
        Все скобки \(>0\), т.к. модули членов ряда убывают. 
        Тогда \(S_{2k} > 0\) и возрастает с увеличением \(k\).
        \begin{equation}
            0 < S_{2k} \leq U_{1}
        \end{equation}
        Сумма \(S_{2k}\) положительна и ограничена сверху, 
        тогда она имеет конечный предел \(S\). \\
        Рассмотрим сумму \(S_{2k+1} = S_{2k} + U_{2k+1}\):
        \begin{equation}
            \begin{align*}
                &\lim_{k\to\infty} S_{2k+1} = \lim_{k\to\infty} + \underbrace{\lim_{k\to\infty}U_{2k+1}}_{\text{= 0 т.к. ряд убывает}} = S+0= S\\
                &\lim_{n\to\infty}S_{n} = S.
            \end{align*}
        \end{equation}
        \(\implies\) ряд сходится.
    \end{proof}

    \begin{corollary}
        
    \end{corollary}
\end{thm}



\section{Абсолютная сходимость знакопеременного ряда}
\begin{opr}
    Ряд называется \textbf{абсолютно сходящимся}, если сходится ряд модулей его членов.
    Если ряд сходится, а ряд модулей его членов расходится, то такой ряд называется 
    \textbf{условно сходящимся}
\end{opr}

\begin{thm}
    Если ряд сходится абсолютно, то он сходится.\\
    \begin{proof}
        \(\sum_{n=1}^{\infty}U_{n}\) знакопеременный, сходится абсолютно.
        \(\sum_{n=1}^{\infty}V_{n} = U_{n} + |U_{n}|\).
        Очевидно, что \(V_{n} < 2|U_{n}|\). 
        Так как сходится \(\sum_{n=1}^{\infty} |U_{n}|\), 
        сходится и \(\sum_{n=1}^{\infty} 2|U_{n}|\).
        Тогда: \(V_{n}\) сходится по мажорантному признаку,
        \begin{equation}
            \begin{align*}
                &\sum_{n=1}^{\infty}V_{n} = \sum_{n=1}^{\infty}(U_{n} + |U_{n}|) = \sum_{n=1}^{\infty}U_{n} + \sum_{n=1}^{\infty}|U_{n}|\\
                &\implies \sum_{n=1}^{\infty}U_{n} = \underbrace{\sum_{n=1}^{\infty}V_{n}}_{\text{сход.}} - \underbrace{\sum_{n=1}^{\infty}|U_{n}|}_{\text{сход.}} \implies \sum_{n=1}^{\infty}U_{n} \text{ сходится}
            \end{align*}
        \end{equation}
    \end{proof}
\end{thm}

\subsection{Свойства абсолютно сходящихся рядов}
\end{document}