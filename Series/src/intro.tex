\documentclass[11pt,a4paper]{article}

\usepackage[utf8]{inputenc}
\usepackage[russian]{babel}
\usepackage{setspace}
\usepackage{amsmath, amssymb, amsthm}
\usepackage[breaklinks]{hyperref}

\setlength{\textheight}{24cm}
\setlength{\textwidth}{16cm}
\setlength{\topmargin}{-5mm}
\setlength{\oddsidemargin}{0cm}
\setlength{\evensidemargin}{0cm}
\setstretch{1.3}

\sloppy

\relpenalty=10000
\binoppenalty=10000

\theoremstyle{definition}
\newtheorem{definition}{Опр.}[section]

\renewcommand{\thedefinition}{\arabic{definition}}

\date{}

\begin{document}

\begin{definition}
Сумма членов бесконечной последовательности $\{U_n\}_{n=1}^{\infty}$ называется \textbf{рядом} и равна пределу его частичной суммы:
    \begin{equation}
        U_1 + U_2 + U_3 + \dots = \sum_{n=1}^{\infty} U_n = \lim_{n \to \infty} \sum_{k=1}^{n} U_k
    \end{equation}
При этом $U_n$ называется \textbf{общим членом последовательности}.
\end{definition}

\end{document}
