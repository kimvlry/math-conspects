\documentclass[11pt,a4paper]{article}
\usepackage[utf8]{inputenc}
\usepackage[T2A]{fontenc}

\usepackage[russian]{babel}
\usepackage{setspace}
\usepackage{amsmath, amssymb, amsthm}
\usepackage[breaklinks]{hyperref}
\usepackage{xcolor}
\usepackage{lipsum}
\usepackage{fancyhdr}
\usepackage{geometry}
\usepackage{microtype}
\sloppy


\geometry{top=2cm, bottom=2cm, left=2cm, right=2cm}
\pagestyle{fancy}
\fancyhf{} 
\fancyhead[R]{\vspace{0.5em}4 сентября 2024\vspace{-0.5em}}


\setlength{\parindent}{0pt}
\setlength{\parskip}{1em}
\setlength{\textheight}{24cm}
\setlength{\textwidth}{16cm}
\setlength{\topmargin}{-5mm}
\setlength{\oddsidemargin}{0cm}
\setlength{\evensidemargin}{0cm}
\setstretch{1.3}
\singlespacing

\sloppy

\relpenalty=10000
\binoppenalty=10000

\theoremstyle{definition}
\newtheorem{definition}{Опр.}
\newtheorem{sv}{Свойство}
\newtheorem{sled}{Следствие}
\newtheorem{priz}{Признак}

\newenvironment{myproof}[1][\proofname]{
  \noindent \textbf{#1}
  \begin{flushleft}
    \hspace{-1em}$\square$
  \end{flushleft}
}{
  \hfill $\blacksquare$
}


\renewcommand{\thedefinition}{\arabic{definition}}




\begin{document}
\begin{definition}
Сумма TESTчленов бесконечной последовательности $\{U_n\}_{n=1}^{\infty}$ называется \textbf{рядом} и равна пределу его частичной суммы:
    \begin{equation}
        U_1 + U_2 + U_3 + \dots = \sum_{n=1}^{\infty} U_n = \lim_{n \to \infty} \sum_{k=1}^{n} U_k
    \end{equation}
При этом $U_n$ называется \textbf{общим членом последовательности}.
\end{definition}

\par
\begin{definition}
    Остаток ряда $r_n$ — сумма, остающаяся после отбрасывания частичной суммы ряда. 
\end{definition}

\par

\textbf{Исследование сходимости ряда:}\\
предел последовательности частичных сумм $\exists$ и конечен $\implies$ ряд сходится (далее — СХ), \\
иначе ряд расходится (далее — РХ).

\textbf{Знакоположительные ряды. Признаки сходимости}

\begin{priz}
    Признак сравнения с неравенством (мажорантный)\\
    \begin{equation}
        \begin{aligned}
        &\text{Для рядов } \sum_{n = 1}^{\infty} a_n \text{ и } \sum_{n = 1}^{\infty} b_n: \forall n : a_n \leq b_n \implies \\
        &\sum_{n = 1}^{\infty} b_n \text{ сходится,} \implies \sum_{n = 1}^{\infty} a_n \text{ тоже сходится;} \\
        &\sum_{n = 1}^{\infty} a_n \text{ расходится,} \implies \sum_{n = 1}^{\infty} b_n \text{ тоже расходится.}
        \end{aligned}
    \end{equation}

    \begin{myproof}
        \text{TODO}
    \end{myproof} 
\end{priz}

\begin{priz}
    Предельный (относительный) признак
    \begin{equation}
        \sum_{n = 1}^{\infty}a_n, \sum_{n = 1}^{\infty}b_n : \exists\lim_{n\to\infty}\frac{a_n}{b_n} = L. \text{Тогда ряды одновременно СХ или РХ}.\\ 
    \end{equation} 
    
    \begin{myproof}
        \text{TODO}
    \end{myproof} 
\end{priz}


\par
\textbf{Свойства сходящихся рядов}

\begin{sv}
    $\sum U_n $ СХ $\implies r_n\to 0$ 
\end{sv}


\begin{sv}
    Необходимое условие сходимости\\
    $\sum{U_n}$ сходится $\implies$ $U_n \to 0$ при $n\to \infty$ . 

    \begin{myproof}
    \begin{equation}
        \lim_{n\to\infty} U_n 
        = \lim_{n\to\infty} (S_n - S_{n - 1})
        = \lim_{n\to\infty}S_n - \lim_{n\to\infty} S_{n-1} 
        = S - S 
        = 0
    \end{equation}
\end{myproof}

NB: свойство не работает в обратную сторону. Пример - гармонический ряд: 
    \begin{equation}
        \begin{aligned}
            &1 + \underbrace{\frac{1}{2}}_{\text{}} + \underbrace{\frac{1}{3} + \frac{1}{4}}_{\text{}} + \underbrace{\frac{1}{5} + \frac{1}{6} + \frac{1}{7} + \frac{1}{8}}_{\text{}} + \cdots + \frac{1}{n} = S_n \\
            &1 + \frac{1}{2} + \frac{1}{4} + \frac{1}{4} + \frac{1}{8} + \frac{1}{8} + \frac{1}{8} + \frac{1}{8} + \cdots + \frac{1}{n} = S_n \\ \\
            &S'_n \leq S_n \\\\
            &S'_1 = 1 \\
            &S'_2 = 1 + \frac{1}{2} \\
            &S'_4 = 1 + 2 \cdot \frac{1}{2} \\
            &S'_8 = 1 + 3 \cdot \frac{1}{2} \\
            &S'_{16} = 1 + 4 \cdot \frac{1}{2} \\
            &\dots \\
            &S'_{2^k} = 1 + k \cdot \frac{1}{2} \xrightarrow{\text{при $k\to\infty$}} \infty 
        \end{aligned} 
    \end{equation}
    Если сумма \( S^{'}_n \) меньшего ряда расходится, то сумма \( S_n \) также расходится. При этом общий член \( U_n \) стремится к нулю:
    \[
    U_n = \frac{1}{n} \to 0 \text{ при } n \to \infty.
    \]

\end{sv}
\par
\begin{sv}
    Достаточное условие расходимости\\
    \(U_n\ \not\to 0\) при \(n\to\infty\) \(\implies\) РХ (предположим, что СХ. Тогда $U_n\to 0$ по св-ву 2.)

\end{sv}
\par
\begin{sv}
    Линейность\\
    - сходящиеся ряды можно почленно складывать, сходимость рез-та не изменится\\
    - при умножении всех членов ряда на скаляр сходимость не меняется (а сумма ряда изменится в это число раз)\\
    - сходимость не меняется для частичной суммы/остатка ряда
\end{sv}




\end{document}
