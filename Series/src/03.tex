\documentclass[11pt,a4paper]{article}
\usepackage[utf8]{inputenc}
\usepackage[T2A]{fontenc}

\usepackage[russian]{babel}
\usepackage{setspace}
\usepackage{amsmath, amssymb, amsthm}
\usepackage[breaklinks]{hyperref}
\usepackage{xcolor}
\usepackage{lipsum}
\usepackage{fancyhdr}
\usepackage{geometry}
\usepackage{microtype}
\sloppy


\geometry{top=2cm, bottom=2cm, left=2cm, right=2cm}
\pagestyle{fancy}
\fancyhf{} 
\fancyhead[R]{\vspace{0.5em}4 сентября 2024\vspace{-0.5em}}


\setlength{\parindent}{0pt}
\setlength{\parskip}{1em}
\setlength{\textheight}{24cm}
\setlength{\textwidth}{16cm}
\setlength{\topmargin}{-5mm}
\setlength{\oddsidemargin}{0cm}
\setlength{\evensidemargin}{0cm}
\setstretch{1.3}
\singlespacing

\sloppy

\relpenalty=10000
\binoppenalty=10000

\theoremstyle{definition}
\newtheorem{definition}{Опр.}
\newtheorem{sv}{Свойство}
\newtheorem{sled}{Следствие}
\newtheorem{priz}{Признак}

\newenvironment{myproof}[1][\proofname]{
  \noindent \textbf{#1}
  \begin{flushleft}
    \hspace{-1em}$\square$
  \end{flushleft}
}{
  \hfill $\blacksquare$
}


\renewcommand{\thedefinition}{\arabic{definition}}




\title{Функциональные ряды}
\date{02.10.24}

\begin{document}
\maketitle

\begin{opr}
    Ряд, члены которых - это функции
    \begin{equation}
        \sum_{n = 1}^{\infty} U_{n}(x)
    \end{equation}
    где \(\{U_{n}(x)\}\) называется функциональной последовательностью 
\end{opr}

\begin{ex}
    \begin{equation}
        \begin{align*}
            &\sum_{n = 1}^{\infty} a_{n}x^{n} \text{  - степенной ряд}
        \end{align*}
    \end{equation}
\end{ex}

\begin{opr}
    \textbf{Частичная сумма функционального ряда}
    \begin{equation}
        S_{n}(x) = \sum_{k = 1}^{n}U_{k}(x)
    \end{equation}
    \(\{S_{n}(x)\}\) - последовательность частичных сумм 
\end{opr}

\begin{opr}
    \textbf{Функциональная последовательность}
    \begin{equation}
        \{f_{n}(x)\}
    \end{equation}
\end{opr}



\section{Сходимость функциональной последовательности}

\begin{opr}
    сходимость функциональной последовательности:
    Если \(f(x), f_{n}(x)\) определены на \(X\):
    \begin{equation}
        \begin{align*}
            &\{f_{n}(x)\} \text{ сходится на } X \text{ к } f(x) 
            \implies\lim_{n\to\infty} f_{n}(x) = f(x).\\
            &f_{n}(x)\to f(x) \Leftrightarrow \forall x_{0}\in X\\
            &\lim_{n\to\infty} f_{n}(x) = f_{x}
        \end{align*}
    \end{equation}
\end{opr}

\begin{opr}
    Если при фиксированном \(x_{0}\) функциональный ряд сходится, тоговорят, что 
    он сходится в точке \(x = x_{0}\). При этом множество \(X\) всех таких точек 
    назыается \textbf{областью сходимости} ряда. Аналогично для расходимости. 
\end{opr}

\begin{thm}
    \textbf{Остаток сходящегося функционального ряда равен 0}.
    Если ряд сходится к сумме \(S(x)\) и \(S(x) = S_{n}(x) + r_{n+1}(x)\) 
    (\(r\) - остаток ряда), то
    \begin{equation}
        \lim_{n\to\infty}r_{n+1}(x) 
        = \lim_{n\to\infty}(S(x) - S_{n}(x)) 
        = S(x) - \underbrace{\lim_{n\to\infty}S_{n+1}(x)_{= S(x)}} = 0
    \end{equation}
\end{thm}

\begin{ex}
    \begin{equation}
        \begin{align*}
            &\sum \frac{n + 1}{x^{3}} \text{  везде расходится:}\\
            &\lim_{n\to\infty} U_{n}(x) = \lim_{n\to\infty} \frac{n+1}{x^{3}} 
            = \frac{1}{x^{3}}\lim_{n\to\infty}(n+1) = \infty .
        \end{align*}
    \end{equation}
\end{ex}

\begin{ex}
    \begin{equation}
        \begin{align*}
            &\sum x^{n} \cdot n! \text{  - сходится только в 0}\\
            &\sum \frac{x^{n}}{n!} \text{  - везде сходится}\\
            &\sum \frac{1}{x^{n}\cdot n!} \text{  - расходится при \(x = 0\)}
        \end{align*}
    \end{equation}
\end{ex}

\begin{ex}
    \begin{equation}
        \begin{align*}
            &\sum x^{n} - \text{  геом. ряд}\\
            & \abs{x} < 1\implies \text{  сходится}\\
            & \abs{x} >= 1\implies \text{  расходится}\\
            & (-1; 1) \text{  - область сходимости}
        \end{align*}
    \end{equation}
\end{ex}


\section{Нахожедение области абсолютной сходимости}
\begin{thm}
    \textbf{}
    \begin{equation}
        \lim_{n\to\infty} \abs{\frac{U_{n+1}(x)}{U_{n}(x)}} = q(x) \hspace{0.5cm}
        \begin{cases}
            q(x) = 0 \implies \text{  сходится на \(-\infty; +\infty\)}\\
            q(x) = \infty \implies \text{  сходится только в \(x=0\)}\\
            \text{иначе: интервал из н-ва } \abs{q(x)} < 1. \text{ Затем исследуются концы интервала}
        \end{cases}
    \end{equation}
\end{thm}


\begin{ex}
    \begin{equation}
        \begin{align*}
            &\sum \frac{(-1)^{n}\cdot 5^{n}}{(2n+5)(3x - 2)^{2n}}\\
            &\\
            &U_{n+1}(x) = \frac{5\cdot 5^{n}}{(2n+5)(3x - 2)^{2n + 2}}\\
            &\frac{U_{n+1}(x)}{U_{n}(x)} = \dots = \frac{5(2n+3)}{(2n+5)(3x - 2)^{2}}\\
            &\lim_{n\to\infty} \abs{\frac{5(2n+3)}{(2n+5)(3x - 2)^{2}}} = \frac{5}{(3x - 2)^{2}} \underbrace{\lim_{n\to\infty} \frac{2n+3}{2n+5}}_{\text{\(=1\)}} < 1\\
            &\frac{5}{(3x - 2)^{2}} < 1\\
            &\frac{(3x - 2)^{2}}{5} > 1\\
            &(3x - 2)^{2} > 5\\
            &\abs{3x - 2} > \sqrt{5}\\
            & 3x - 2 > \sqrt{5} \implies x > \frac{2 + \sqrt{5}}{3}\\
            & 3x - 2 < - \sqrt{5} \implies x < \frac{2 - \sqrt{5}}{3}\\
            &x\in (-\infty; \frac{2 - \sqrt{5}}{3}) \cup (\frac{2 + \sqrt{5}}{3}; +\infty) \text{  - область абсолютной сходимости ряда}
        \end{align*}
    \end{equation}
    границы проверяем отдельно, исследуем возникающие числовые ряды:
    \begin{proof}
        \color{YellowOrange}\textbf{TODO}
    \end{proof}
\end{ex}

\begin{thm}
    \textbf{признак Коши}
    \begin{equation}
        \lim_{n\to\infty} \abs{\sqrt[n]{U_{n}(x)}} < 1\implies \text{  решаем}
    \end{equation}
\end{thm}



\section{Равномерная сходимость функционального ряда}
\begin{opr}
    ряд называется равномерно сходящимся на множестве 
    \(X \Leftrightarrow \forall \epsilon > 0: \space \exists N(\epsilon)\) 
    и не зависящее от \(x\):
    \begin{equation}
        \begin{align*}
            &\forall n > N \implies \abs{S_{n}(x) - S(x)} < \epsilon \Leftrightarrow \abs{R_{n}(x)} < \epsilon\\
            &\abs{S_{n}(x) - S(x)} = \abs{S(x) - S_{n}(x)} = \abs{R_{n}(x)}
        \end{align*}
    \end{equation}
\end{opr}

\begin{ex}
    ряд, сходящийся неравномерно:
    \begin{equation}
        \begin{align*}
            \color{YellowOrange}\textbf{TODO}
        \end{align*}
    \end{equation}
\end{ex}

\subsection{Признаки равномерной сходимости}

\begin{thm}
    Признак равномерной сходимости \textbf{Вейрштрасса}. 
    Если члены функционального ряда \(\sum U_{n}(x)\) 
    не превосходят на некотором множестве \(X\) членов сходящегося числового ряда,
    то на \(X\) ряд \(\sum U_{n}(x)\) сходится равномерно

    \begin{proof}
        \begin{equation}
            \begin{align*}
                &1. \sum U_{n}(x) & & \text{Остаток: }R_{n}\\
                &2. \sum \abs{U_{n}(x)} & & \text{Остаток: }F_{n}\\
                &3. \sum C_{n} & \text{ числовой ряд, сходится. }& \text{Остаток: }P_{n}\\
                &\\
                &\forall x\in X: \abs{U_{n}(x)} \leq C_{n}\\
            \end{align*}
        \end{equation}
        Если ряд 3 сходится, то ряд 2 сходится на \(X\) по признаку сравнения. 
        Тогда ряд 1 сходится абсолютно на \(X \implies\) сходится на \(X\).\\
        Докажем равномерную сходимость. 
        \begin{equation}
            \begin{align*}
                &U_{n}(x) \leq \abs{U_{n}(x)} \leq C_{n}\\
                &R_{n}(x) \leq F_{n}(x) \leq P_{n}(x) \\
                &\abs{R_{n}(x)} \leq F_{n}(x) \leq P_{n}(x) \forall x\in X
            \end{align*}
        \end{equation}
        Так как 3 сходится, то \(\lim_{n\to\infty} P_{n} = 0\) 
        (по св-ву сходящегося числового ряда). Тогда
        \begin{equation}
            \begin{align*}
                &\Leftrightarrow \forall\epsilon > 0: \exists N(\epsilon)\\
                &\implies \abs{P_{n} - 0} < \epsilon \\
                &\implies \abs{P_{n} < \epsilon};\\
                &\abs{P_{n} < \epsilon}\forall x\in X
            \end{align*}
        \end{equation}
    \end{proof}
\end{thm}
\end{document}