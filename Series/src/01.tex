\documentclass[11pt,a4paper]{article}
\usepackage[utf8]{inputenc}
\usepackage[T2A]{fontenc}

\usepackage[russian]{babel}
\usepackage{setspace}
\usepackage{amsmath, amssymb, amsthm}
\usepackage[breaklinks]{hyperref}
\usepackage{xcolor}
\usepackage{lipsum}
\usepackage{fancyhdr}
\usepackage{geometry}
\usepackage{microtype}
\sloppy


\geometry{top=2cm, bottom=2cm, left=2cm, right=2cm}
\pagestyle{fancy}
\fancyhf{} 
\fancyhead[R]{\vspace{0.5em}4 сентября 2024\vspace{-0.5em}}


\setlength{\parindent}{0pt}
\setlength{\parskip}{1em}
\setlength{\textheight}{24cm}
\setlength{\textwidth}{16cm}
\setlength{\topmargin}{-5mm}
\setlength{\oddsidemargin}{0cm}
\setlength{\evensidemargin}{0cm}
\setstretch{1.3}
\singlespacing

\sloppy

\relpenalty=10000
\binoppenalty=10000

\theoremstyle{definition}
\newtheorem{definition}{Опр.}
\newtheorem{sv}{Свойство}
\newtheorem{sled}{Следствие}
\newtheorem{priz}{Признак}

\newenvironment{myproof}[1][\proofname]{
  \noindent \textbf{#1}
  \begin{flushleft}
    \hspace{-1em}$\square$
  \end{flushleft}
}{
  \hfill $\blacksquare$
}


\renewcommand{\thedefinition}{\arabic{definition}}




\date{4.09.24, 18.09.24}
\title{Знакоположительные ряды. \\Свойства сходящихся рядов}
\begin{document}
\maketitle

\begin{opr}
    \textbf{Ряд}\\
    Сумма членов бесонечной последовательности
    \(\{a_{n}\}^{\infty}_{n=1}\) называется рядом
    и равна пределу последовательности 
    его частичной суммы:
    \begin{equation}
        a_{1} + a_{2} + a_{3} ... 
        = \sum_{n=1}^{\infty} U_{n}
        = \lim_{n\to\infty}\sum_{k=1}^{n} U_{k}
    \end{equation}
    При этом \(a_{n}\) называется \textbf{общим членом ряда}
\end{opr}

\begin{opr}
    \textbf{Частичная сумма ряда} - это сумма его первых \(n\) членов:
    \begin{equation}
        S_{n} = \sum_{k = 1}^{n} U_{k}
    \end{equation}
\end{opr}

\begin{opr}
    \textbf{Остаток ряда} \(r_{n}\) - сумма ряда, остающаяся
    после отбрасывания частичной суммы ряда.
\end{opr}

\begin{remark}
    Ряд и его остаток сходятся или расходятся одновременно.
    \begin{proof}
        \color{YellowOrange}\textbf{TODO}
        % \begin{equation}
        %     \begin{align*}
        %         &\sum_{n=1}^{\infty} U_{n} \text{ сходится } 
        %         \Leftrightarrow \exists\lim_{n\to\infty}\sum_{k=1}^{\n}U_{k}\in\R\\
        %         &
        %     \end{align*}
        % \end{equation}
    \end{proof}
\end{remark}

\section{Исследование сходимости ряда}
\begin{theorem}
    \begin{equation}
        \begin{align*}
            \text{существует } 
            \lim_{n\to\infty}\sum_{k=1}^{n} U_{k} = S\in\R
            &\implies \text{ряд сходится ("к сумме \(S\)")} \\
            \text{иначе} 
            &\implies \text{ряд расходится}
        \end{align*}
    \end{equation}
\end{theorem}



\subsection{Признаки сходимости для знакоположительных рядов}
\begin{priz}
    \label{priz:maj} \textbf{Мажорантный признак}\\
    Для рядов \(\sum_{n=1}^{\infty} a_{n}\) 
    и \(\sum_{n=1}^{\infty} b_{n} : 
    \forall n: a_{n} \leq b_{n} \implies\) 

    \begin{equation}
        \begin{align*}
            \sum_{n=1}^{\infty} a_{n} \text{  расходится} 
            &\implies \sum_{n=1}^{\infty} b_{n} \text{  тоже расходится}\\
            \sum_{n=1}^{\infty} b_{n} \text{  сходится} 
            &\implies \sum_{n=1}^{\infty} a_{n} \text{  тоже сходится.}\\
        \end{align*}
    \end{equation}

    \\

    \begin{proof}
        Пусть \(a_{1} + a_{2} + \dots + a_{n} = S_{a}\) - частичная сумма первого ряда,\\
        \(b_{1} + b_{2} + \dots + b_{3} = S_{b}\) - частичная сумма второго ряда.
        Т.к. \(a_{n}\leq b_{n}\implies S_{a}\leq S_{b}\). \\
        Пусть 2-й ряд сходится, тогда 
        \begin{equation}
            \(\exists\lim_{n\to\infty} S_{b} = L_{b}\),
        \end{equation}
        причем \(S_{b} < L_{b}\implies S_{a} \leq S_{b} < L_{b}\implies |S_{a}|\) ограничена сверху числом \(L_{b}\), \\
        тогда \(\sum_{n=1}^{\infty} a_{n}\) сходится.\\
        \newline
        Пусть 1-й ряд расходится, тогда \(\lim S_{a} = \infty\).Т.к. \(S_{a} \leq S_{b}\), \(\lim_S_{b} = \infty\),\\
        тогда \(\sum_{n=1}^{\infty} a_{b}\) расходится.\\
    \end{proof}
    \textbf{Эталонные ряды:}
    Гармонический ряд
    \color{YellowOrange}\textbf{TODO}
\end{priz}


\begin{priz}
    \label{priz:pred} \textbf{Предельный признак} \\
    Ряд  \(\sum_{n=1}^{\infty} a_{n}\)  исследуемый, 
    \(\sum_{n=1}^{\infty} b_{n}\)  - вспомогательный
    \begin{equation}
        \begin{align*}
            \sum_{n=1}^{\infty} a_{n}, \sum_{n=1}^{\infty} b_{n}:
            \exists\lim_{n\to\infty}\frac{a_{n}}{b_{n}} = L,\\
            \text{Тогда ряды \textbf{одновременно} сходятся или расходятся}
        \end{align*}
    \end{equation}

    \begin{proof}
        \begin{equation}
            \begin{align*}
                &\lim_{n\to\infty}\frac{a_{n}}{b_{n}} = L \\
                &\Longleftrightarrow \forall \epsilon > 0 : \exists N(\epsilon): \forall n > N \implies |\frac{a_{n}}{b_{n}} - L| < \epsilon\\
                &\Longleftrightarrow -\epsilon < \frac{a_{n}}{b_{n}} - L < \epsilon \implies L > 0 \implies L - \epsilon < \frac{a_{n}}{b_{n}} < L + \epsilon\\
                &\Longleftrightarrow b_{n}(L - \epsilon) < a_{n} < b_{n}(L + \epsilon)\\
                \\
            \end{align*}
        \end{equation}
        Пусть сходится ряд \(b_{n}, \implies\) сходится \(\sum_{n=1}^{\infty}(L + \epsilon)b_{n} 
        \implies \sum_{n=1}^{\infty}a_{n}\) сходится по мажорантному \hyperlink{priz:maj}{пр.}
        Пусть расходится ряд \(b_{n} \implies\) расходится \(\sum_{n=1}^{\infty}(L - \epsilon)b_{n} 
        \implies \sum_{n=1}^{\infty}a_{n}\) расходится по мажорантному \hyperlink{priz:maj}{пр.}   
        \\    
    \end{proof}
    \textbf{Эталонные ряды:}\\
    \color{YellowOrange}\textbf{TODO}
\end{priz}


\begin{priz}
    \label{priz:dalamber} \textbf{Признак Д'Аламбера} \\
    \begin{equation}
        \begin{align*}
            \lim_{n\to\infty}\frac{a_{n+1}}{a_{n}} = L\\
            &L < 1\implies \text{ряд сходится}\\
            &L > 1\implies \text{ряд расходится}\\
            &L = 1 \text{ряд может сходиться или расходится}
        \end{align*}
    \end{equation}
    \begin{proof}
        \begin{equation}
            \begin{align*}
                &\lim_{n\to\infty}\frac{a_{n+1}}{a_{n}} = L \\
                &\Longleftrightarrow \forall \epsilon > 0 \hspace{0.5cm} \exists N(\epsilon) : 
                n>N\implies |\frac{a_{n + 1}}{a_{n}} - L| < \epsilon\\
                &\Longleftrightarrow L - \epsilon < \frac{a_{n + 1}}{a_{n}} < L + \epsilon\\
                &\\
                &\frac{a_{n+1}}{a_{n}} = q  < L + \epsilon\\
            \end{align*}
        \end{equation}
        Докажем сходимость при \(L < 1\).\\
        Подберем \(\epsilon\) так, чтобы \(L+\epsilon < 1\) и \(q < 1\):
        \begin{equation}
            \begin{align*}
                &a_{2} = q a_{1}\\
                &a_{3} = q^{2} a_{1}\\
                &a_{4} = q^{3} a_{1}\\
                & \dots \\
                & \sum_{n=1}^{\infty} a_{n} = \sum_{n=1}^{\infty} q^{n} a_{1} = a_{1}\sum_{n=1}^{\infty}q^{n}.
            \end{align*}
        \end{equation}
        Если \(q < 1\), то ряд сходится. Для \(q > 1\) аналогично
    \end{proof}
\end{priz}


\begin{priz}
    \label{priz:radcauchy} \textbf{Радикальный признак Коши} \\
    \begin{equation}
        \begin{align*}
            \lim_{n\to\infty}\sqrt[n]{U_{n}} = L \\
            &L < 1\implies \text{ряд сходится}\\
            &L > 1\implies \text{ряд расходится}\\
            &L = 1\text{ряд может сходиться или расходится}
        \end{align*}
    \end{equation}
    \begin{proof}
        \begin{equation}
            \begin{align*}
                &\lim_{n\to\infty}\sqrt[n]{U_{n}} = L\\
                &\Longleftrightarrow\forall \epsilon > 0 
                \hspace{0.5cm} \exists N(\epsilon): \forall n > N(\epsilon)
                \implies |\sqrt[n]{U_{n}}| < \epsilon.\\
                &L - \epsilon < \sqrt[n]{U_{n}} < L + \epsilon\\
                &\sqrt[n]{U_{n}} = q < 1.
            \end{align*}
        \end{equation}
        Пусть \(L > 1\). Выберем \(\epsilon: L - \epsilon > 1 \implies q > 1\)\\
        \(U_{n} = q^{n} > 1 \implies\lim U_{n} \leq 0\), т.е. ряд расходится.
    \end{proof}
\end{priz}


\begin{priz}
    \label{priz:intcauchy} \textbf{Интегральный признак Коши} \\
    Дан ряд \(\sum_{n=1}^{\infty} U_{n}\), члены которого являются значениями 
    непрерывной, неотрицательной и монотонно \textbf{убывающей} 
    на \([1, +\infty]\) функции \(f(x)\) при целых \(x\):
    \begin{equation}
        U_{1} = f(1), U_{2} = f(2), \dots , U_{n} = f(n).
    \end{equation}
    Тогда ряд \(\sum_{n=1}^{\infty} U_{n}\) 
    и несобственный интеграл \(\int_{1}^{+\infty} f(x)dx \)
    сходятся или расходятся \textbf{одновременно}

    \begin{proof}
        \color{YellowOrange}\textbf{TODO}
    \end{proof}
\end{priz}


\subsection{Свойства сходящихся рядов}
\begin{sv}
    \label{sv:neob}
    Необходимое условие сходимости ряда
    \begin{equation}
        \sum_{n=1}^{\infty} a_{n} \text{сходится}
        \implies \lim_{n\to\infty} a_{n} = 0
    \end{equation}

    \begin{proof}
        \color{YellowOrange}\textbf{TODO}
    \end{proof}

    \begin{remark}
        свойство не работает в обратную сторону. \\
        Пример - гармонический ряд (расходится): 
        \begin{equation}
            \begin{align*}
                1 
                + \underbrace{\frac{1}{2}}_{\text{}} 
                + \underbrace{\frac{1}{3} + \frac{1}{4}}_{\text{}} 
                + \underbrace{\frac{1}{5} + \frac{1}{6} + \frac{1}{7} + \frac{1}{8}}_{\text{}}
                + ... 
                + \frac{1}{n}
                = S_{n}
                \\
                1
                + \frac{1}{2}
                + \frac{1}{4} + \frac{1}{4}
                + \frac{1}{8} + \frac{1}{8} + \frac{1}{8} + \frac{1}{8}
                + ... 
                + \frac{1}{n}
                = S^{'}_{n}
                \\
                \\
                &S^{'}_{n} \leq S_{n}\\
                &S^{'}_{1} = 1\\
                &S^{'}_{2} = 1 + \frac{1}{2}\\
                &S^{'}_{4} = 1 + \frac{1}{2}\cdot 2\\
                &S^{'}_{8} = 1 + \frac{1}{2}\cdot 3\\
                &...
                \\
                &S^{'}_{2^{k}} = 1 + \frac{1}{2}\cdot 2^{k} \to\infty \text{ при } k\to\infty
            \end{align*}
        \end{equation}
        Если сумма \(S^{'}_{n}\) меньшего ряда расходится, то сумма \(S_{n}\) также расходится (по \hyperlink{priz:maj}{мажорантному признаку}). 
        При этом общий член стремится к нулю:
        \( \frac{1}{n}\to\infty \text{ при } n\to\infty \)
    \end{remark}
\end{sv}

\begin{sv}
    Достаточное условие расходимости ряда 
    \begin{equation}
        \lim_{n\to\infty} a_{n} \nrightarrow 0 
        \implies \text{ ряд расходится}
    \end{equation}
    \begin{proof}
        Предположим, что сходится. 
        Тогда \(\lim_{n\to\infty} a_{n} = 0\) по \hyperlink{sv:neob}{св-ву 1.2.1}, противоречие
    \end{proof}
\end{sv}

\begin{sv}
    Сходящиеся ряды можно почленно складывать, при этом сходимость не изменится,
    а сумма полученного ряда будет равна сумме сумм исходных рядов.
    \begin{proof}
        \color{YellowOrange}\textbf{TODO}
        % \begin{equation}
        %     \begin{align*}
        %         1. 
        %         &\sum_{n=1}^{\infty} U_{n} = S_{n}^{1}; \hspace{1cm} \lim_{n\to\infty}S_{n} = S^{1} \\
        %         2.
        %         &\sum_{n=1}^{\infty} c\cdot U_{n} = S_{n}^{2} , \hspace{1cm} \lim_{n\to\infty}S_{n} = S^{2}
        %     \end{align*}
        % \end{equation}
    \end{proof}
\end{sv}

\begin{sv}
    При умножении сходящегося ряда на число, сходимость не изменятся, при этом
    сумма изменится в это число раз.
    \begin{proof}
        \begin{equation}
            \begin{align*}
                1. 
                &\sum_{n=1}^{\infty} U_{n} = S_{n}; \hspace{1cm} \lim_{n\to\infty}S_{n} = S \\
                2.
                &\sum_{n=1}^{\infty} c\cdot U_{n} = S_{n}^{'} ,
            \end{align*}
                
            \begin{align*}
                \text{тогда: }\hspace{0.5cm} 
                 S_{n}^{'} = \sum_{n=1}^{\infty} c\cdot U_{n} = c \sum_{n=1}^{\infty} U_{n} = c\cdot S_{n}, 
                \hspace{0.5cm} 
                \lim_{n\to\infty} S_{n}^{'} = c\lim_{n\to\infty} S_{n} = c\cdot S
            \end{align*}
        \end{equation}
    \end{proof}
\end{sv}

\begin{sv}
    Сходимость ряда не изменится, если к нему прибавить или отбросить конечное 
    число слагаемых.
\end{sv}
\end{document}