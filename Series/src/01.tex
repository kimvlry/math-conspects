\documentclass[11pt,a4paper]{article}
\usepackage[utf8]{inputenc}
\usepackage[T2A]{fontenc}

\usepackage[russian]{babel}
\usepackage{setspace}
\usepackage{amsmath, amssymb, amsthm}
\usepackage[breaklinks]{hyperref}
\usepackage{xcolor}
\usepackage{lipsum}
\usepackage{fancyhdr}
\usepackage{geometry}
\usepackage{microtype}
\sloppy


\geometry{top=2cm, bottom=2cm, left=2cm, right=2cm}
\pagestyle{fancy}
\fancyhf{} 
\fancyhead[R]{\vspace{0.5em}4 сентября 2024\vspace{-0.5em}}


\setlength{\parindent}{0pt}
\setlength{\parskip}{1em}
\setlength{\textheight}{24cm}
\setlength{\textwidth}{16cm}
\setlength{\topmargin}{-5mm}
\setlength{\oddsidemargin}{0cm}
\setlength{\evensidemargin}{0cm}
\setstretch{1.3}
\singlespacing

\sloppy

\relpenalty=10000
\binoppenalty=10000

\theoremstyle{definition}
\newtheorem{definition}{Опр.}
\newtheorem{sv}{Свойство}
\newtheorem{sled}{Следствие}
\newtheorem{priz}{Признак}

\newenvironment{myproof}[1][\proofname]{
  \noindent \textbf{#1}
  \begin{flushleft}
    \hspace{-1em}$\square$
  \end{flushleft}
}{
  \hfill $\blacksquare$
}


\renewcommand{\thedefinition}{\arabic{definition}}




\date{4 сентября 2024}
\begin{document}
\maketitle

\begin{opr}
    \textbf{Ряд}\\
    Сумма членов бесонечной последовательности
    \(\{a_{n}\}^{\infty}_{n=1}\) называется рядом
    и равна пределу последовательности 
    его частичной суммы:
    \begin{equation}
        a_{1} + a_{2} + a_{3} ... 
        = \sum_{n=1}^{\infty} U_{n}
        = \lim_{n\to\infty}\sum_{k=1}^{n} U_{k}
    \end{equation}
    При этом \(a_{n}\) называется \textbf{общим членом ряда}
\end{opr}

\begin{opr}
    Остаток ряда \(r_{n}\) - сумма ряда, остающаяся
    после отбрасывания частичной суммы ряда.
\end{opr}

\section{Исследование сходимости ряда}
\begin{opr}
    \begin{equation}
        \begin{align*}
            \text{существует и конечен } 
            \lim_{n\to\infty}\sum_{k=1}^{n} U_{k} 
            &\implies \text{ряд сходится} \\
            \text{иначе} 
            &\implies \text{ряд расходится}
        \end{align*}
    \end{equation}
\end{opr}



\subsection{Признаки сходимости для знакоположительных рядов}
\begin{priz}
    Мажорантный признак\\
    Для рядов \(\sum_{n=1}^{\infty} a_{n}\) 
    и \(\sum_{n=1}^{\infty} b_{n} : 
    \forall n: a_{n} \leq b_{n} \implies\) 

    \begin{equation}
        \begin{align*}
            \sum_{n=1}^{\infty} a_{n} \text{  расходится} 
            &\implies \sum_{n=1}^{\infty} b_{n} \text{  тоже расходится}\\
            \sum_{n=1}^{\infty} b_{n} \text{  сходится} 
            &\implies \sum_{n=1}^{\infty} a_{n} \text{  тоже сходится.}
        \end{align*}
    \end{equation}
    \begin{proof}
        TODO
    \end{proof}
\end{priz}

\begin{priz}
    Предельный признак \\
    \begin{equation}
        \begin{align*}
            \sum_{n=1}^{\infty} a_{n}, \sum_{n=1}^{\infty} b_{n}:
            \exists\lim_{n\to\infty}\frac{a_{n}}{b_{n}} = L,\\
            \text{Тогда ряды \textbf{одновременно} сходятся или расходятся}
        \end{align*}
    \end{equation}

    \begin{proof}
        TODO
    \end{proof}
\end{priz}



\subsection{Свойства сходящихся рядов}
\begin{sv}
    \label{sv:neob}
    Необходимое условие сходимости ряда
    \begin{equation}
        \sum_{n=1}^{\infty} a_{n} \text{сходится}
        \implies \lim_{n\to\infty} a_{n} = 0
    \end{equation}
    \begin{remark}
        не работает в обратную сторону. 
        Пример - гармонический ряд: \(\lim_{n\to\infty}a_{n} = 0\), 
        но ряд расходится
    \end{remark}
\end{sv}

\begin{sv}
    Достаточное условие расходимости ряда 
    \begin{equation}
        \lim_{n\to\infty} a_{n} \nrightarrow 0 
        \implies \text{ ряд расходится}
    \end{equation}
    \begin{proof}
        Предположим, что сходится. 
        Тогда \(\lim_{n\to\infty} a_{n} = 0\) по \hyperlink{sv:neob}{св-ву 1.2.1}, противоречие
    \end{proof}
\end{sv}

\begin{sv}
    Линейность. \\
    - сходящиеся ряды можно почленно складывать, сходимость рез-та не изменится\\
    - при умножении всех членов ряда на число сходимость не меняется (а сумма ряда изменится в это число раз)\\
    - сходимость не меняется для частичной суммы/остатка ряда
\end{sv}

\end{document}